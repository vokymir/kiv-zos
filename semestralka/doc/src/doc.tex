\documentclass[czech, kiv, he, sem, pdf, viewonly]{fasthesis}

\usepackage{pdfpages}

\title{Souborový systém založený na i-uzlech}
\author{Jakub}{Vokoun}{}{}

\nobastardtitle
\nocopyrightnotice

\begin{document}
\frontpages[notm]
\tableofcontents

\chapter{Zadání}
Vytvořte zjednodušený \term{souborový systém} založený na \term{i-uzlech}.
Implementujte jednoduché \term{TUI} pro práci se souborovým systémem
které umožní uživateli používat konkrétní příkazy s definovaným chováním.
Bonusový úkol pro studenta s orion loginem začínajícím na písmeno \textbf{j}.

Kompletní znění zadání včetně popisu všech příkazů je na straně: \pageref{app:assignment}.

\chapter{Popis implementace}
Program je napsán v \term{c++ 23} s využitím standardní knihovny a knihovny
\command{readline} která slouží k uchování historie příkazů za běhu programu.

Kompletní implementace souborového systému se nachází ve třídě \command{Filesystem}.
Tato třída nabízí metody pro práci se systémem, jak elementární funkčnost (například
\command{inode_create}) tak komplexnější (například \command{file_create}),
které vhodně spojují funkčnost elementárních funkcí do uceleného celku
metody vyšší úrovně abstrakce.

Při změně velikosti souboru (zvětšování, jelikož zmenšování podle zadaných
příkazů není třeba, kromě zmenšení na 0, tedy mazání souboru) je zajištěna atomicita
všech operací, jelikož se jedná o častou operaci, která má velkou šanci neúspěchu.
Využívá se rozdělení klastrů na režijní a datové. Datové jsou ty, které přímo obsahují
data souboru, zatímco režijní jsou nepřímé prvního a druhého řádu, obsahující odkazy na
datové respektive režijní klastry. Nejprve se předalokuje dostatek klastrů jak pro
data tak režii - tím se eliminuje většina problémů pramenících z nedostatku
volného místa. Následuje zápis režijních klastrů prvního řádu do klastrů druhého řádu,
po čemž se zapíší datové klastry do režijních prvního řádu. Až po provedení těchto
operací se zapíše nová velikost do i-uzlu. Konzistence dat je zajištěna dopředným
načtením všech existujících klastrů a jejich zápis ve stejném pořadí.

Všechny příkazy které se dají v TUI použít mají vlastní třídu, které je oddědena
od třídy \command{ICommand} s virtuální metodou \command{execute_inner} a
společnou metodou \command{execute}, která volá virtuální, odchytává jakékoliv
\term{výjimky} a vypisuje standardní zprávy definované v zadání,
popisující úspěch nebo chybu při běhu příkazu.


\chapter{Uživatelská příručka}
Program je třeba zkompilovat a to buď pomocí standardních \command{cmake} příkazů,
anebo přiloženého \term{bash skriptu} \command{auto/build.bash}, který vhodně
aplikuje standardní cmake na zvolenou adresářovou strukturu.
Pro kompilaci nejspíš lze použít gcc, autorem byl použit \command{clang 20.1.8}.

Spustitelný soubor lze spusit standardně, bude se ale nacházet v adresáři
\command{bin}. Pro jednoduchost lze použít přiložený skript \command{auto/run.bash},
který se zároveň postará o zadání jména souboru, ve kterém se souborový systém nachází.

Při spuštění lze specifikovat příznak \command{-v} nebo \command{--vocal},
které při běhu programu vypisují další diagnostické informace.

Po spuštění programu lze zadávat všechny příkazy specifikované zadáním.
Navíc byly přidány dva další: \command{help}, který vypisuje seznam všech
použitelných příkazů a \command{exec}, který umožňuje spouštět \term{shell}
příkazy. Tento příkaz byl přidán z důvodu snadného testování.

Každý příkaz lze zavolat s příznakem \command{-h} po kterém se místo provedení
příkazu zobrazí nápověda k jeho používání.


\chapter{Závěr}
Souborový systém je ve funkčním stavu s uspokojivou úrovní funčnosti. Před
rozšiřováním funkcionality by bylo záhodno se zaměřit na fungování třídy
\command{Filesystem}, konkrétně na:

\begin{itemize}
    \item použití návrhového vzoru \term{jedináček}.
    Toto rozhodnutí výrazně urychlilo vývoj, ale zhodnotit užitečnost tohoto rozhodnutí
    je zcela na místě.
    \item dekompozici třídy. Ta jako taková zastává více povinností, než by
    bylo moudré. Dekomponovat by šlo horizontálně (rozdělit na třídy, které se zaměřují
    pouze na práci s i-uzly, klastry,...) anebo vertikálně (rozdělit na nízkourovňovou
    funkcionalitu, tj. práci s i-uzly, klastry, a vysokoúrovňovou, tj. práci se soubory
    a adresáři).
    \item práci s režijními klastry - unifikovat logiku tak, aby bylo jednoduché
    přidávat nepřímé režijní klastry vyššího řádu, bez nutné úpravy kódu.
\end{itemize}

Kromě této hlavní třídy by měl být zrevidován způsob emulování terminálu, přesunutí
funkcionality do třídy ze souboru \command{main.cpp}.

\appendix

\chapter{Kompletní zadání}\label{app:assignment}
Následující strany obasahují kompletní a nezkrácené zadání semestrální práce.

\includepdf[pages=-, scale=0.75, pagecommand={}, frame=true]{../ZOS2025_SP_01.pdf}

\backmatter
\backpage
\end{document}
