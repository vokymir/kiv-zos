\documentclass[czech, kiv, he, sem, pdf, viewonly]{fasthesis}

\usepackage{pdfpages}

\title{Souborový systém založený na i-uzlech}
\author{Jakub}{Vokoun}{}{}

\nobastardtitle
\nocopyrightnotice

\begin{document}
\frontpages[notm]
\tableofcontents

\chapter{Zadání}
Vytvořte zjednodušený \term{souborový systém} založený na \term{i-uzlech}.
Implementujte jednoduché \term{TUI} pro práci se souborovým systémem
které umožní uživateli používat konkrétní příkazy s definovaným chováním.
Bonusový úkol pro studenta s orion loginem začínajícím na písmeno *j*.

Kompletní znění zadání včetně popisu všech příkazů na straně: \pageref{app:assignment}.


\chapter{Analýza úlohy}
Souborový systém má být uložen jako jeden soubor v jeho binární reprezentaci.
Běžné rozdělení takového souboru pro souborové systémy založené na i-uzlech:

\begin{itemize}
    \item \verb{Super blok} obsahuje informace o struktuře souboru

    \item \verb{Bitmapa i-uzlů} informace o obsazenosti i-uzlů
    \item \verb{Bitmapa klastrů} informace o obsazenosti datových bloků, klastrů
    \item \verb{I-uzly}
    \item \verb{Klustry}
\end{itemize}

Pro implementaci bude třeba následujících struktur,
které se budou binárně ukládat přímo do souborového systému:

\begin{itemize}
    \item \command{superblock} bude obsahovat kritické informace o souborovém systému.
    Protože bude obsahovat adresy všech ostatních bloků
\end{itemize}


\chapter{Popis implementace}

    \section{Souborový systém}
    Kompletní implementace souborového systému se nachází ve třídě \command{Filesystem}.



    \section{TUI}


\chapter{Uživatelská příručka}


\chapter{Závěr}

\appendix

\chapter{Kompletní zadání}\label{app:assignment}
Následující strany obasahují kompletní a nezkrácené zadání semestrální práce.

\includepdf[pages=-, scale=0.75, pagecommand={}, frame=true]{../ZOS2025_SP_01.pdf}

\backmatter
\backpage
\end{document}
